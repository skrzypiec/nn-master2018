\documentclass[]{article}
\usepackage{mathtools}
\usepackage[pdftex]{graphicx}	
\usepackage{amsmath,amsfonts,amsthm}	

\newtheorem{theorem}{Theorem}[section]
\newtheorem{lemma}[theorem]{Lemma}
\newtheorem{proposition}[theorem]{Proposition}
\newtheorem{corollary}[theorem]{Corollary}

\theoremstyle{definition}
\newtheorem{definition}{Definition}[section]



%opening
\title{Neural network is universal approximator}
\author{}

\begin{document}

\maketitle
Plan:
\begin{itemize}
	\item wstęp o sigmoidzie wraz z grafiką
	\item definicje i twierdzenia (hahn-banach, riesz)
	\item twierdzenie i dowód o gęstości kombinacji liniowej sigmoid
	\item dowód graficzny
	\item cytowania
\end{itemize}


\subsubsection{}

Neural networks with sigmoidal activation functions can approximate to arbitrary accuracy any functional continuous mapping from one finite-dimensional space to another, provided the number N of hidden units is sufficiently large.

		

wiki: "A sigmoid function is a mathematical function having a characteristic "S"-shaped curve or sigmoid curve. Often, sigmoid function refers to the special case of the logistic function defined by the formula"

$$
\sigma(x) = \frac{1}{1+e^{-x}}
$$


\begin{figure}[h]
	\centering
	\includegraphics[width=0.65\linewidth]{sigmoid}
	%\caption{$\frac{1}{1+e^{-x}}$}
\end{figure}

\newpage

Let $I_n$ denote the n-dimensional unit cube, $[0,1]^n$. The space of continous functions on $I_n$ is denoted by $C(I_n)$ and we use $||f||$ to denote the supremum norm of an $f \in C(I_n)$. The space of finite, signed regular Borel measures on $I_n$ is denoted by $M(I_n)$.



\begin{definition}
	We say that $\sigma$ is sigmoidal if
	\begin{eqnarray*}
		\sigma(x) \rightarrow \begin{cases} 1 \;\;\;\text{as} &x \rightarrow +\infty\\ 0 \;\;\;\text{as} &x \rightarrow -\infty\end{cases}
	\end{eqnarray*}
	
\end{definition}

\begin{definition}
We say that $\sigma$ is discriminatory if for a measure $\mu \in M(I_n)$ 

$$
\int_{I_n} \sigma \left( y^Tx + \theta \right) d\mu(x) = 0
$$
for all $y\in \mathbf{R}$ and $\theta \in \mathbf{R}$ implies that $\mu = 0$.
	
\end{definition}


Hahn-Banach theorem shows how to extend linear functionals from subspaces to whole spaces. Moreover, we can do it in a way that respects the boundedness
properties of the given functional. The most general formulation of the theorem requires a preparation

\begin{definition}
A sublinear functional is a function $f:V \rightarrow \mathbf{R}$ on a vector space $V$ which satisfies subadditivity (1) and positive homogenity conditions (2)
\begin{eqnarray}
f\left(x+y\right) &\leq& f\left(x\right) + f\left(y\right) \;\;\;\;\;\;\;\;\;\;\;\forall x,y  \in V \\
f\left(\alpha x\right) &=&\alpha f\left(x\right) \;\;\;\;\;\;\;\;\;\;\;\;\;\;\;\;\;\;\;\;\; \forall \alpha\geq 0, x \in V
\end{eqnarray}
\end{definition}

\begin{theorem}[Hahn-Banach theorem for real vector spaces]
	If $p : V \rightarrow \mathbf{R}$ is a sublinear function, and $\psi : U \rightarrow \mathbf{R}$ is a linear functional on a linear subspace $U \subset V$, and satisfying $\psi(x) \leq p(x)$ $\forall x \in U$.
	Then there exists a linear extension $\Psi:V \rightarrow \mathbf{R}$ of $\psi$ to the whole space $V$, such that
	
	\begin{itemize}
		\item $\Psi(x) = \psi(x)$ $\forall x \in U$
		\item $\Psi(x) \leq p(x)$ $\forall x \in V$
	\end{itemize}
	
	Rudin 1991, Th 3.2
	
	%Let $V$ be a real vector space and $p : V \rightarrow \mathbf{R}$ a sublinear %functional on $V$. Let $\psi$ be a linear functional defined on a subspace $U %\subset V$, and satisfying $\psi(x) \leq p(x)$ $\forall u \in U$. Then there %exists a linear functional $\Psi:V \rightarrow \mathbf{R}$ such that
\end{theorem}

\begin{theorem}[Riesz representation theorem]
	Let $H$ be a Hilber space over $\mathbf{R}$, and $T$ a bounded linear functional on $H$. If $T$ is a bounded linear functional on a Hilbert space $H$ then there exist some $g \in H$ such that for every $f \in H$ we have (http://www.math.jhu.edu/~lindblad/632/riesz.pdf)
	$$
	T(f) = \langle f,g \rangle \;\;\;\;\;\; \forall f \in H
	$$
	
	Any bounded linear functional T on the space of compactly supported continuous functions on $X$ is the same as integration against a measure $\mu$. (http://mathworld.wolfram.com/RieszRepresentationTheorem.html)
	$$
	Tf = \int f d\mu
	$$
	
\end{theorem}

\begin{theorem}[]
	Let $\sigma$ be any continous discriminatory function. Then finite sums of the form
$$
G\left(x\right) = \sum_{j=1}^{N} \alpha_j \sigma\left(y_j^Tx + \theta_j\right)
$$

are dense in $C(I_n)$. In other words, given any $f \in C(I_n)$ and $\epsilon >0$, there is a sum, $G(x)$, of the above form, for whic

$$
|G(x) - f(x)| < \epsilon \;\;\;\;\;\;\;\; \forall x \in I_n
$$
\end{theorem}

\begin{proof}
Let $S \subset C(I_n)$ be the set of functions of the form $G(x)$. Clearly $S$ is a linear subspace of $C(I_n)$. We claim that the closure of $S$ is all of $C(I_n)$. 

Assume that closure of $S$ is not all of $C(I_n)$. Then the closure of $S$, say $R$, is a closed proper subspace of $C(I_n)$. By the Hahn-Banach theorem, there is a bounded linear functional on $C(I_n)$, call it L, with the property that $L \neq 0$ but $L(R) = L(S) = 0$.

By the Riesz Representation Theorem, this bounded linear functional, L, is of the form 

$$
L(h) = \int_{I_n} h(x)d\mu(x)
$$

for some $\mu \in M(I_n)$, for all $h \in C(I_n)$. In particular, since $\sigma(y^Tx + \theta)$ is in $R$ for all $y$ and $\theta$, we must have that

$$
\int_{I_n} \sigma \left(y^Tx + \theta \right) d\mu(x) = 0
$$

for all $y$ and $\theta$.

However, we assumed that $\sigma$ was discriminatory so that this condition implies that $\mu = 0$ contradicting our assumpition. Hence, the subspace $S$ must be dense in $C(I_n)$.

This demonstrates that sums of the form

$$
G(x) \sum_{j=1}^{N} \alpha_j \sigma\left(y_j^Tx + \theta_j\right)
$$

are dense in $C(I_n)$ providing that $\sigma$ is continous and discriminatory.

\end{proof}


\subsection{visual proof}



\end{document}
